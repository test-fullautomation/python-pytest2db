% Copyright 2020-2023 Robert Bosch GmbH

% Licensed under the Apache License, Version 2.0 (the "License");
% you may not use this file except in compliance with the License.
% You may obtain a copy of the License at

% http://www.apache.org/licenses/LICENSE-2.0

% Unless required by applicable law or agreed to in writing, software
% distributed under the License is distributed on an "AS IS" BASIS,
% WITHOUT WARRANTIES OR CONDITIONS OF ANY KIND, either express or implied.
% See the License for the specific language governing permissions and
% limitations under the License.

\pkg\ is a command-line tool that enables you to import
\href{https://docs.pytest.org/}{pytest} XML result
files into TestResultWebApp's database for presenting an overview about the
whole test execution and detail of each test result.

\begin{figure}[h!]
   \includegraphics[width=1\linewidth]{./pictures/data_flow.png}
   \caption{Tool data flow}
\end{figure}

\pkg\ tool requires serveral arguments, including the location of the pytest
XML result file(s) to parse all information of test execution result and
TestResultWebApp's database credential for importing that result.

\href{https://github.com/test-fullautomation/testresultwebapp}{TestResultWebApp}
requires some mandatory information to manage and display the test result
properly, but the generated \emph{*xml} file contains only the basic test result
information. So that, \pkg\ tool will set those required information with
\hyperref[default-values]{default values}.

Besides, you can also use optional arguments of \pkg\ tool to provide missing
information or you want to overwrite them with the expected values.

Finally, \pkg\ also allows you to append existing results in the database, which
is helpful when you need to update previous test results or add the missing
XML result file(s) from previous tool execution.
